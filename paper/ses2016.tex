\documentclass{fose2016}           % for pLaTeX2e
%\documentclass[english]{fose2016} % for English papers
%\documentclass[ascii]{fose2016}   % for ASCII pTeX 

\usepackage[dvipdfmx]{graphicx}
%\usepackage{epsfig}

\title{依存関係の向きを考慮したファイルの変更時間への影響分析}
\etitle{}% TODO dependency file, vector, develop time, impact analysis
\author{上田 裕己}{Ueda Yuki, 島根大学}
\author{神谷 年洋}{Kamiya Toshihiro, 島根大学}

\begin{document}



\maketitle

\begin{abstract}
%大規模なソフトウェア開発では多くのファイルからソフトウェアが成り立っている.
%そのため,あるファイルを変更した場合ほかのファイルにも影響を及ぼす.

%既存研究では依存関係がソフトウェアの品質や開発者の経験に影響を与えることがわかっている.

%本研究では,変更されたソースコードの依存関係の依存の向きに着目して,依存関係をもつファイルが次に変更されるまでの期間への影響を調査した.
 
%実験結果から変更されているソースコードに依存関係の向きや変更を行った人物によって開発期間に大きく影響を与えることが分かった.


ソースコードの依存関係はソフトウェアの保守プロセスにおいて,ソースコードの同時変更の予測や,
不具合の修正や機能拡張のためにソースコードを変更する際の担当者の割り当てなどに利用されている.
影響波及解析(change impact anlysis)はこのような応用の基礎となっているモデルである.
影響波及解析では、あるソースファイルが変更されると,そのソースファイルに依存している
別のソースファイルも変更が必要となるというモデルであり,
既存研究により実験的な評価が行われている.

本研究では、依存関係の依存の向きに着目し,影響波及解析で仮定されている方向と逆向きの関係も含めて,
あるソースファイルの変更と,依存関係がある他のソースファイルの変更の時間間隔を,
オープンソースプロダクトのリポジトリを対象として調査した.
また,近年の分散開発においてはプロダクトの開発チーム外の
開発者が修正を行う(いわゆる「GitHubのプルリクエスト」)事例もあることに着目し,
変更を行った開発者によって依存関係と変更時間にどのような影響があるかも調べた.

\end{abstract}

\begin{eabstract}

\end{eabstract}

%はじめに
\section{はじめに} 
% 現在開発者の経験を図るために多くの研究で開発者がいままでに変更したファイルが利用されている.
ソフトウェア開発において開発期間の短縮や不具合の回避のために,開発者に対してそれぞれ適した仕事を割り振る研究が行われている.\cite{Bird,Patanamon}
Bird\cite{Bird}らは専門レベルという開発者があるファイルに対してどれだけ詳しいかという度合いによってソフトウェアの故障率に影響を与えるという研究を行った.
これにより,ソフトウェアプロジェクトの管理者は問題に対して最適な開発者を割り当てることができる.
しかし,対象ファイル群が依存する他のファイル群(以降,依存ファイル群)に対する専門レベルは考慮されていない.
このため,変更が行われたファイルと依存関係にあるファイルに対して十分な検査が行えず,不具合を見逃す恐れがある.

影響波及解析はファイルの変更に対して,そのファイルに依存するファイルにも変更が必要になるのか解析するモデルである.
影響波及解析の研究の一つとして,Briand\cite{Briand}らは依存関係が変更までの時間に影響を与えることを調査した.
この既存研究では,変更されたファイルに対して依存しているファイルのみを調査している.
しかし,既存研究では変更されたファイルに対して依存されているファイル,また,依存するファイルに依存するファイルに対して調査が行われていない.
これにより,同じ機能を重複して実装したり, 機能を抽象化する際の変化を追えていない可能性がある.
本研究では影響波及解析で仮定されている方向とは逆の変更されたファイルに依存されているファイルや間接的に依存関係のあるファイルに対しても調査する.

そのため,本論文では変更されたファイルに対して依存関係のあるファイルが変更されるまでの時間の間隔(以降,変更時間間隔(単位:日))への影響について調査を行った.
ある変更が行われてから,早い変更時間間隔で変更が行われるファイルには, 不具合が発生やリファクタリングが起きていると考えられる.
変更時間間隔を調査することにより,どのような依存関係のあるファイルに開発者が注目すればいいのかが,わかることが期待される.

そのために変更が行われたファイルに対して,プロジェクト内のすべてのファイルがどのような依存関係を持っているのか6つに分類を分けそれぞれの変更時間間隔への影響を調査する.
また, プロジェクトに権限を持っている人物による変更であることによる変更時間間隔への影響を調査するために,同じ人物が変更した場合,マージとなった変更の場合でも同様の調査を行った.

本論文では\ref{関連研究}章で関連研究について書き,\ref{ツール・サービス}章では利用したツール・サービスについて,\ref{アプローチ}章で問題へのアプローチ,\ref{調査対象のデータセット}章では調査対象のデータセットについて,\ref{実験}章で実験の内容と結果,\ref{考察}章で実験結果についての考察,\ref{妥当性の検証}章では実験の妥当性についての検証,最後に\ref{まとめ}章でまとめを行う.


%関連研究
\section{関連研究}\label{関連研究}
開発者はソフトウェアに対して多くの変更を行う.
その過程で,変更した部分だけでなく,変更されていない部分へも何らかの影響があることが考えられる.
そこで行われる影響波及解析では変更の影響を受ける部分の識別が行われる.
影響波及解析にはさまざまな手法が提案されているが\cite{Ryder,Kondo},本研究では変更による影響を受けたファイルを特定するため,ファイルと粒度の近いクラス単位での解析\cite{Ryder}を行う.
ソフトウェアの依存関係を評価する方法として,システムのモジュール間での依存関係をマトリクス表示するDSM(Dependency Structure Matrix)がある.\cite{Nord}
本論文ではこの評価方法を利用して依存関係を求める.

% 利用したツール
%「分散バージョン管理システムにおけるバージョンの概念」や「プルリクエスト」を説明するものにしましょう。図1や図2の説明に繋がるように。
\section{利用したツール・サービス}\label{ツール・サービス}
% git
\subsection{バージョン管理システム, git}
バージョン管理システムとはファイルの変更の履歴を保存するためのシステムである.
現在多くのソフトウェアではバージョンコントロール管理システムが利用されている.
今回調査を行ったvert.vプロジェクトも同様にバージョン管理システムの一つであるgitで管理されている.
gitなどの分散型のバージョン管理システムでは,多くの開発者が一つのソフトウェアを変更していても,変更の重複を防ぐことができる.
変更を記録することをコミットを呼び,コミットを利用することで前回のコミットから現在までに変更されたファイルを調べることができる.

gitで利用されているものにブランチがある.
ブランチは履歴の流れを分岐して記録していくためのものであり,同じプロジェクトの中で複数の変更を同時に進めていくことを目的としている.
有益な変更が行われたブランチの変更は主となるブランチであるmasterに統合される.これをマージという.
マージもコミットの一つであり,マージされることによりマージされたブランチにあったコミットはmasterのコミットとしても存在する.

% GitHub
\subsection{GitHub}
さまざまなオープンソースソフトウェアを管理しているサービスである.
これを利用して認可のあるプロジェクトや開発者の状況を確認したり,ほかの開発者と議論をすることができる.

プロジェクト外の開発者もGitHubにあるソースコードを別のブランチとして変更し,その変更を投稿することができる.
この投稿はプルリクエストと呼ばれている.
GitHub内でプルリクエストが投稿された時,コードレビューという,変更を行った開発者とは別の開発者によるコードの査読が行われる.
このコードレビューで有用と判断された変更はmasterにマージされる.


% アプローチ
\section{アプローチ}\label{アプローチ}
変更時間間隔に影響を受ける依存関係,開発者,gitの利用状況についての傾向を調査するために,以下の項目について調査を行う.
\begin{itemize}
\item RQ1:変更されたファイルに依存関係のあるファイルが変更されるまでの時間は依存関係のないファイルより短くなる
\item RQ2:依存関係があるファイル(fがgに依存)の変更において、gが変更されたあとfが変更時間間隔は、fとgの開発者が同じ場合には、異なる場合よりも,短くなる
\item RQ3:依存関係があるファイル(fがgに依存)の変更において、gが変更されたあとfが変更時間間隔は、fのコミットが複数のブランチのマージとなったコミットである場合には,異なる場合よりも,長くなる
\end{itemize}
% RQ1
\subsection{RQ1:変更されたファイルに依存関係のあるファイルが変更されるまでの時間は依存関係のないファイルより短くなる}
依存関係に種類を定義することによってどの種類がファイルの変更時間間隔に影響を与えるのか調査を行った..

ここでは,以下のことについて調査を行う.
\begin{enumerate}
\item RQ1-1ファイルfがファイルgに依存するとき、gが変更されたあとfが変更されるまでの時間は、依存関係がないファイルと比較して短い
\item RQ1-2 ファイルfがファイルgに依存するとき、fが変更されたあとgが変更されるまでの時間は、依存関係がないファイルと同じである
\end{enumerate}

また,本研究ではソースコードのうち以下のような記述を依存関係として扱う.
変更が行われたファイルに対して以下の例のような記述が存在した場合,クラスAを定義しているファイルに依存しているとする.

\begin{itemize}
\item 参照による代入や呼び出し,定義(例:ClassA a = new ClassA())
\item 継承 (例: ClassB extends ClassA)
\end{itemize}

\subsubsection{依存関係の分類}
依存関係があるファイルにも依存関係の方向や間接的な依存関係をもつものを考慮するためにプロジェクト内のすべてのファイルを表\ref{tab:依存関係の分類}に示したような分類に分ける.
本論文ではこの分類を依存関係の分類と呼ぶことにする.図\ref{fig:dependency}に依存関係の分類についての概要を示した.
変更が行われた時,プロジェクト内のファイルは少なくとも一つの依存関係の分類に属することとなる.

\begin{table}
\caption{依存関係の分類}
\begin{tabular}{|l|r|r|} \hline
分類名 & 概要 \\ \hline
root & 変更されたファイル \\ \hline
dependee & rootが依存するファイル \\ \hline
depender & rootが依存されるファイル \\ \hline
dependee2 & dependeeが依存するファイル \\ \hline
depender2 & dependerが依存されるファイル \\ \hline
other & 上記に当てはまらないファイル \\ \hline
\end{tabular}
\label{tab:依存関係の分類}
\end{table}

\begin{figure}[t]
\centering
\includegraphics[width=\columnwidth]{dependency.pdf}
\caption{依存関係の分類図}
\label{fig:dependency} 
\end{figure}

% RQ2
\subsection{RQ2:依存関係があるファイル(fがgに依存)の変更において、gが変更されたあとfが変更時間間隔は、fとgの開発者が同じ場合には、異なる場合よりも、短くなる}
同一人物が変更を行った場合とそうでない場合でRQ1と同じく変更時間間隔への影響を調査した.
これを調べることにより, 依存関係による影響を解決している人物はどのような依存関係に集中しているのか,
また,同じ開発者であることによって,どのような依存関係のあるファイルが影響を受けるかをRQ1と同様に調査する.

% RQ3
\subsection{RQ3:依存関係があるファイル(fがgに依存)の変更において、gが変更されたあとfが変更時間間隔は、fのコミットが複数のブランチのマージとなったコミットである場合には,異なる場合よりも,長くなる}
マージとなるコミットの場合はプルリクエストが投稿された後,コードレビューがされているコミットである可能性が高い.
しかし,マージされていないコミットの場合はコードレビューが行われていないことが考えられる.
レビューが行われていないファイルの多くはほかの開発者に確認をされていないため信頼性が低いと考えられる.
よってマージとなるコミットはそうでないコミットよりも変更時間間隔が長くなると考えた.

ここではマージされたコミット変更ではそうでない変更の変更時間間隔を,RQ1と同様に依存関係の分類別に,比較してどのような影響があるのか調査をする.

                                                                                                                                                                                                                                                                                                                                                                                                                                                                                                                                                                                                                                                                                                                                                                                                                                                                                                                                                                                                                                                                                                                                                                                                                                                                                                                                                                                 % 調査方法
\section{調査対象のデータセット}\label{調査対象のデータセット}
GitHub上のeclipseプロジェクトの中で最もstar数が多かったvert.vプロジェクトを調査,
vert.vプロジェクトは2013年から2016年まで変更され続け,合計2000以上のコミットが行われている.
eclipseプロジェクトには多くのオープンソースソフトウェアがあり,今後ほかのプロジェクトと比較するのに適していると考えた

% コミットの分類
\subsection{コミットの分類}
本研究ではコミットの分類を以下の4つに分けた.

\begin{enumerate}
\item masterのコミット \label{enum:basedcommit}
\item masterのマージとなるコミット \label{enum:mergecommit}
\item マージされたブランチでmasterに\ref{enum:basedcommit}として取り込まれたコミット \label{enum:mergedbranch}
\item マージされていないブランチで行われたコミット
\end{enumerate}

図\ref{fig:gitimage} にコミットの分類を図示する.

\begin{figure}[t]
\centering
\includegraphics[width=\columnwidth]{git_image.pdf}
\caption{gitによるコミットの分類分け}
\label{fig:gitimage} 
\end{figure}

本研究で分析するコミットは\ref{enum:basedcommit}と\ref{enum:mergecommit}のコミットである.
\ref{enum:mergedbranch}のコミットはマージされ\ref{enum:basedcommit}としてmasterに存在するため,\ref{enum:basedcommit}のコミットのみを分析する.

% データセット・取得データ
\subsection{取得データ}
csv形式で時系列とともに出力する.
出力するデータを表\ref{tab:初期データセット}に示した.


\begin{table}[htb]
\caption{出力するデータセット}
\begin{tabular}{|l|r|r|} \hline
属性名 & 概要 \\ \hline
filepass & ファイルの絶対パス \\ \hline
commit\_no & HEADからのコミットの距離 \\ \hline
date & 変更が行われた日付 \\ \hline
author & 変更を行った開発者の名前 \\ \hline
kind & 依存関係の種類 \\ \hline
is\_merge & マージされた後のコミットであるかどうか \\ \hline
\end{tabular}
\label{tab:初期データセット}
\end{table}

filepassとは出力するファイルのファイルパスである.
同一のcommitNoに対してその時点でのプロジェクトでのすべてのjavaファイルを出力している.
このファイルパスは一意ではなく,同じfilepassとcommitNoでもkindが別のものがある.
データ型は文字列型である.

commit\_noとは最新の変更からの変更の距離を示している.データ型は数値型である.
dateとは変更が行われた日付を出力している.データ型は日付型である.
authorとは変更を行った開発者の名前を出力している.データ型は文字列型である.
kindとは表{tab:依存関係の分類}で示した依存関係の分類を出力している.データ型は文字列型である.
is\_mergeとはコミットがmergeされたコミットかどうかを分類しているものであり,論理型の値を得る.
図\ref{fig:gitimage}で示されているとおり,\ref{enum:basedcommit}と\ref{enum:mergecommit}のコミットを識別している.


これらの情報をもとに変更時間間隔を求める.
データ型は数値型である.
このデータはkindがrootのもの以外に対してつけられる.

%実験
\section{実験}\label{実験}

%実験・手順
\subsection{手順}
以下の手順でデータ取得を行う.
\begin{enumerate}
\item プロジェクトの状態を一つ過去のものに戻す.
\item その状態のプロジェクト内のファイルの情報,をXML形式で出力する.
\item 生成されたXMLファイルから依存関係を求める.
\item その状態のプロジェクト内のすべてのファイルの情報,表\ref{tab:初期データセット}をcsvファイルに出力する.
\item すべての過去の変更を調べるまで1に戻る
\end{enumerate}
これにより得られたデータ量は141MGとなった.

% doxygen
\subsection{doxygen}
ソースコード解析ツール.XML形式でファイルごとに宣言や状況などを出力することができる.
今回の研究では依存関係を検出するために利用した.


\subsection{実験結果}
依存関係の分類別,また同じ開発者が変更を行っているか,マージとなるコミットであるかによって依存関係を持つファイルの変更時間間隔を出力した.

%実験・RQ1
\subsubsection{RQ1:変更されたファイルに依存関係のあるファイルが変更されるまでの時間は依存関係のないファイルより短くなる}
図\ref{fig:subdate}はそれぞれ依存関係別にみた依存関係の検出から変更時間間隔を箱ひげ図で出力したものである.  

\begin{figure}
\centering
\includegraphics[width=0.5\columnwidth]{date.pdf}
\caption{依存関係の分類ごとの変更時間間隔}
\label{fig:subdate} 
\end{figure}


%ここでのp-value検定の表もだす


図\ref{fig:subdate}から依存関係別で優位に差があることが判明した.
よって以下のRQ2,RQ3でも,この依存関係の分類を用いることにする.


%実験・RQ2
\subsubsection{RQ2:依存関係があるファイル(fがgに依存)の変更において、gが変更されたあとfが変更時間間隔は、fとgの開発者が同じ場合には、異なる場合よりも、短くなる}
依存関係をもつファイルを同じ開発者が変更した場合の変更時間間隔への影響を調査した.
図の\label{fig:author_true_subdate} は同じ開発者が変更した場合の変更時間間隔を出力している.
また,図の\label{fig:author_false_subdate} は異なる開発者が変更した場合の変更時間間隔を出力している.

\begin{figure}
\centering
\begin{minipage}{0.49\columnwidth}
\centering
\includegraphics[width=\columnwidth]{author_date_TRUE.pdf}
\caption{同じ開発者の依存関係の分類ごとの変更時間間隔}
\label{fig:author_true_subdate} 
\end{minipage}
\begin{minipage}{0.49\columnwidth}
\centering
\includegraphics[width=\columnwidth]{author_date_FALSE.pdf}
\caption{異なる開発者の依存関係の分類ごとの変更時間間隔}
\label{fig:author_false_subdate} 
\end{minipage}
\end{figure}


%実験・RQ3
\subsubsection{RQ3:依存関係があるファイル(fがgに依存)の変更において、gが変更されたあとfが変更時間間隔は、fのコミットが複数のブランチのマージとなったコミットである場合には,異なる場合よりも,長くなる}
マージとなるコミットの変更時間間隔の影響を調査した.
図の,\label{fig:merge_true_subdate} はマージとなるコミットの変更時間間隔を出力している.
また,図の\label{fig:merge_false_subdate} はマージではないコミットの変更時間間隔を出力している.

\begin{figure}
\centering
\begin{minipage}{0.49\columnwidth}
\centering
\includegraphics[width=\columnwidth]{merge_date_TRUE.pdf}
\caption{マージとなるコミットの依存関係の分類ごとの変更時間間隔}
\label{fig:merge_true_subdate} 
\end{minipage}
\begin{minipage}{0.49\columnwidth}
\centering
\includegraphics[width=\columnwidth]{merge_date_FALSE.pdf}
\caption{マージではないコミットの依存関係の分類ごとの変更時間間隔}
\label{fig:merge_false_subdate}
\end{minipage}
\end{figure}


%考察
\section{考察}\label{考察}

%考察 RQ1
\subsection{RQ1:変更されたファイルに依存関係のあるファイルが変更されるまでの時間は依存関係のないファイルより短くなる}
図\ref{fig:subdate}について考察する.

最も開発期間が長いものがotherであることから依存関係があるものは依存関係の向きや長さを問わず早く変更される傾向にあると考えられる.
\subsubsection{RQ1-1ファイルfがファイルgに依存するとき、gが変更されたあとfが変更されるまでの時間は、依存関係がないファイルと比較して短い}
dependerやdepender2からわかるように変更されたファイルに依存しているファイルが早く変更される傾向にあることが分かった.
すべてのファイルの変更されるまでの日付の中央値は25日なのに対してdependerは約5日で変更される.
このことから依存関係の距離よりも依存関係の向きが開発機関に影響を与えることが分かった.
基本的に変更によって影響を受けるものは変更されたファイルに依存関係をもつファイルであることが予測できるため,妥当な結果であるといえる.

\subsubsection{RQ1-2 ファイルfがファイルgに依存するとき、fが変更されたあとgが変更されるまでの時間は、依存関係がないファイルと同じである}
しかしdependeeやdependee2もotherよりは開発期間が短くなることから, 変更されたファイルに対して依存されているファイルも影響を受けることが分かった.
変更する理由としては変更後に同じ機能を実装した場合や必要なデータが増えたことが予測される.
自然言語処理やコードクローンの調査を行うことによってファイルや変更の種類について発見が期待される.

以上から,影響があるのは順に依存関係の向き,依存関係の距離が関係していることがわかる.


% 考察 RQ2
\subsection{RQ2:依存関係があるファイル(fがgに依存)の変更において、gが変更されたあとfが変更時間間隔は、fとgの開発者が同じ場合には、異なる場合よりも、短くなる}
図\ref{fig:author_true_subdate} と\ref{fig:author_false_subdate} について考察する

同じ人物が変更した場合と比較し,違う人物が変更したファイルは全体的に中央値が高い結果になっている.
dependee2とotherを比較した場合,違い人物が変更した場合には大きな差がないのに対して,同じ人物が変更した場合は20日近く差があることが分かった.

同じ人物が変更した場合に中央値が下がることに関して考えられることは,同じ開発者が変更したコミットの大部分は同時に変更をしたものである.
多くの開発者が不具合がでないように他の依存関係のあるファイルを変更するので妥当な結果といえる.

別の開発者が変更をした場合にはdependee2とotherに対して大きく差がでる.
これは同じパッケージに入っているために影響がでることが考えられる.

%考察 RQ3
\subsection{RQ3:依存関係があるファイル(fがgに依存)の変更において、gが変更されたあとfが変更時間間隔は、fのコミットが複数のブランチのマージとなったコミットである場合には,異なる場合よりも,長くなる}
図\ref{fig:merge_true_subdate} と\ref{fig:merge_false_subdate} について考察する

マージされたコミットほど変更されるタイミングが遅くなる.
また,マージされていないコミットではdependee2とotherとの差がないことがわかった.


mergeされたコミットの場合mergeされていないコミットと比較して次に変更される時期が遅くなる傾向がある.
以下はその理由を考察した.
\begin{enumerate}
\item レビューを通してから変更を適用するために信頼性が高く,変更の必要がなくなる
\item mergeされ終わり, 作業が減る
\item 変更を行ったのがコアコミッターではないため不具合があったとしても修正に時間がかかる
\end{enumerate}

特にmergeされたコミットではdependee2はotherと中央値が近い数値を出している.
以下に理由について考察した.

\begin{enumerate}
\item 外部からの変更ではあまり抽象的になりがちなファイルを変更しづらい
\item 絶対数が少ないため, 変更されにくい
\end{enumerate}

% 妥当性の検証
\section{妥当性の検証}\label{妥当性の検証}
次に,妥当性について検証を行う.

\subsection{単一のプロジェクトでの検証}
今回の実験で用いたプロジェクトはオープンソースである,eclipseのプロジェクトである.
ほかのオープンソースプロジェクトや商業ソフトウェアで,同様結果が得られるとは限らない.
特にソフトウェアの規模によっては開発者の違いや全体的な開発期間に影響があることが予測される.

\subsection{依存関係の扱いについての影響}
依存関係についての調査では継承や参照もすべて同じ依存関係であるとした.
依存関係の種類\cite{小谷}別に調査することによって変更時間間隔に影響がでる可能性がある.

\subsection{開発者の専門性についての考慮}
Patanamonは開発者の経験についてレビュー・開発経験についての影響について考慮した研究を行った\cite{Patanamon2}.
同じ開発者やマージされた変更であっても開発者の経験によって変更時間間隔に影響が考えられる.

\subsection{ほかのメトリクスとの比較}
影響波及解析の研究として他に依存関係の構造をもとにしたPropagation Cost(PC)を図る研究などもある.\cite{Nord}
このような過去のメトリクスとの比較をすることで本研究の有効性を理解することができるのではないかと考えている.

\subsection{dependeeとdependerの関係}
DSMではファイルがお互いに依存しあうことも考慮されている.
そのため本研究では変更されたファイルが依存するファイルであるdependeeと変更されたファイルが依存されるファイルであるdependeeが互いに同じ要素を持っていることが考えられる.
dependeeはdependerにも含まれる要素の影響のみでotherとの差が生まれていることも考えられる.

% まとめ
\section{まとめ} \label{まとめ}
本研究では,ファイルの依存関係が変更時間間隔に与える影響について調査した.
その結果,依存関係の向きや距離によって変更時間間隔が変化し,
同一の開発者によって変更が行われた場合のようにどのようなファイルでも変更時間間隔が短くなる要素と
,マージではないコミットの場合に依存関係のあるファイルに対して特に変更時間間隔が短くなる要素があることが分かった.

ここから,マージではないコミットほど早く変更されやすいことから,レビューを行わない変更の場合は特に依存関係に注意をする必要がある.
今回は一つのオープンソースプロジェクトのみを調査した.ほかのプロジェクトでも同じような傾向がみられるか調査する必要がある.

\section*{謝辞}
本研究はJSPS科研費16K12412の助成を受けたものである.


\begin{thebibliography}{10}
\bibitem{Ryder}
Ryder, Barbara G., and Frank Tip. "Change impact analysis for object-oriented programs." Proceedings of the 2001 ACM SIGPLAN-SIGSOFT workshop on Program analysis for software tools and engineering. ACM, 2001.
\bibitem{Briand}
Briand, L. C., Labiche, Y., O’Sullivan, L., and Sówka, M. M. (2006). Automated impact analysis of UML models. Journal of Systems and Software, 79(3), 339-352.
\bibitem{Canfora}
 Canfora, G., Ceccarelli, M., Cerulo, L., and Di Penta, M. (2010, September). Using multivariate time series and association rules to detect logical change coupling: An empirical study. In Software Maintenance (ICSM), 2010 IEEE International Conference on (pp. 1-10). IEEE.
\bibitem{Budgen}
Budgen, D., Burn, A. J., Brereton, O. P., Kitchenham, B. A., and Pretorius, R. (2011). Empirical evidence about the UML: a systematic literature review. Software: Practice and Experience, 41(4), 363-392.
\bibitem{Bird}
Bird, Christian, et al. "Don't touch my code!: examining the effects of ownership on software quality." Proceedings of the 19th ACM SIGSOFT symposium and the 13th European conference on Foundations of software engineering. ACM, 2011.
\bibitem{Balachandram}
V. Balachandran: Reducing Human Effort and Improving Quality in Peer Code Reviews using Automatic Static Analysis and Reviewer Recommendation, ICSE'13, pp. 931-940, 2013.
\bibitem{Thongtanunam}
Patanamon Thongtanunam, Chakkrit Tantithamthavorn, Raula Gaikovina Kula, Norihiro Yoshida, Hajimu Iida, Ken-ichi Matsumoto: Who Should Review My Code? A File Location-Based Code-Reviewer Recommendation Approach for Modern Code Review, 22nd IEEE International Conference on Software Analysis, Evolution, and Reengineering, 2015.
\bibitem{Thongtanunam2}
Patanamon Thongtanunam and Shane McIntosh and Ahmed E. Hassan and Hajimu Iida: Revisiting Code Ownership and Its Relationship with Software Quality in the Scope of Modern Code Review, Proc. of the International Conference on Software Engineering (ICSE), 2016
\bibitem{Kotani}
小谷正行, and 落水浩一郎. "UML 記述の変更波及解析に利用可能な依存関係の自動生成." (2008).
\bibitem{Kondo}
近藤和弘, 大畑文明, and 井上克郎. "オブジェクト指向プログラム変更時の影響波及解析手法の提案." 電子情報通信学会技術研究報告. SS, ソフトウェアサイエンス 101.628 (2002): 33-40.
\bibitem{Patanamon} 
Patanamon Thongtanunam, etc...
Who Should Review My Code? A File Location-Based Code-Reviewer Recommendation Approach for Modern Code Review
\bibitem{Patanamon2}
Patanamon Thongtanunam1, Shane McIntosh, Ahmed E. Hassan, Hajimu Iida Revisiting Code Ownership and its Relationship with Software Quality in the Scope of Modern Code Review
\bibitem{Nord}
Nord, Robert L., et al. "Variations on using propagation cost to measure architecture modifiability properties." Software Maintenance (ICSM), 2013 29th IEEE International Conference on. IEEE, 2013.


\end{thebibliography}

\end{document}
